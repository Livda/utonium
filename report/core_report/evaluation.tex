\chapter{Évaluation}

Pendant le développement de notre programme, nous avons utilisé nos propres appels aux différents compilteurs que nous utilisons. Il s'agit de petits
tests qui assurent que notre programme se déroule bien et que les fonctions basique de notre programme soient stables. Cela nous a permis d'optimiser
notre temps en écrivant à la fois les tests de notre logiciel et une évaluation de notre logiciel. Nous avons aussi décider de tester le projet 
d'autres étudiants, Moolinet. Nous avons donc écris des cas de tests dans le langage qu'ils ont défini pour tester la compilation de ceux-ci l'éxecution
dans leur fuzzer.


Les performances de notre testeur vont dépendre quasiement uniquement des appels aux différents compilateurs, le point
qui peut prendre le plus de temps est la compilation et l'éxecution du code. Mais sachant que la pluaprt du temps,
on cherche à garder des tests les plus petits possibles, pour cibler le plus précisment un bug, les temps s'éxecution des programmes est 
relativment faible.

Le passage à l'échelle n'est pas pris en compte dans notre solution, mais il pourrait
l'être si l'on parallélise les appels système. Ainsi, un cluster de calculateurs
pourraient chacun tester ses fonctions sans impacter les autres tests. Notre
solution utilisant des processus déjà existants, il faut juste rendre ces processus
passables à l'échelle.
